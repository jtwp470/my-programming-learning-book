\documentclass[11pt,a4j]{jarticle}
% 利用するパッケージの選択
\usepackage{amsmath,amssymb}
\usepackage{bm}
\usepackage[dvipdfmx]{graphicx}
\usepackage{ascmac}
\usepackage{color}
% 余白の設定
\usepackage[top=20truemm,bottom=20truemm,left=10truemm,right=10truemm]{geometry}

% listings.styの設定 要jlistings.sty
\usepackage{listings,jlisting}
\usepackage{courier}
\lstset{
  language={Caml},% 使用言語
  basicstyle={\ttfamily},% 書体の設定
  identifierstyle={\small},% 
  commentstyle={\small\itshape\color[cmyk]{1,0.4,1,0}},% 注釈の書体
  keywordstyle={\small\bfseries\color[cmyk]{0,1,0,0}},% キーワードの書体
  ndkeywordstyle={\small},% 
  stringstyle={\small\ttfamily\color[rgb]{0,0,1}},
  frame={single},
  breaklines=true,% 行が長い時の改行
  columns=[l]{fullflexible},%
  numbers=left,% 
  xrightmargin=0zw,%
  xleftmargin=3zw,%
  numberstyle={\scriptsize},%
  stepnumber=1,% 行番号増分
  numbersep=1zw,
  lineskip=-0.5ex,
  showstringspaces=false % 空白に文字を表示させない
}
\usepackage{lastpage}
\usepackage{fancyhdr}

\makeatletter
%%%%%%%%%%%%%%%%%%%%%%%%%%%%%%%%%%%%%%%%%%%%%%%%%%%%%%%%%%%%%%%%
%% 要編集
\title{第7回}
\author{201311367 佐藤良祐}
\西暦
\date{\today}
%%%%%%%%%%%%%%%%%%%%%%%%%%%%%%%%%%%%%%%%%%%%%%%%%%%%%%%%%%%%%%%%
\pagestyle{fancy}

% headers & footers
\lhead{ソフトウェア技法 \@title \@author \@date}
\chead{}
\rhead{}
\lfoot{}
\cfoot{\thepage /\pageref{LastPage}}
\rfoot{}
\renewcommand{\headrulewidth}{0pt}
\renewcommand{\footrulewidth}{0pt}
\makeatother

\begin{document}
\section*{課題7.1}
\subsection*{1}
任意のリストxs, ys, 任意の関数fに対して
\begin{verbatim}
map f (xs @ ys) = (map f xs) @ (map f ys)    
\end{verbatim}
が成り立つことを示せ.
\begin{screen}
\begin{verbatim}
(1) xs = [] のとき  
 map f ([] @ ys) = map f ys
                 = [] @ map f ys  (mapの定義)

(2) xsのとき与えられた式が成り立つと仮定してx::xsのときを示す.
 map f ((x::xs) @ ys) = map f (x::(xs @ ys))
                      = f x :: map f (xs @ ys)
                      = f x :: (map f xs) @ (map f ys)
                      = (map f (x::xs)) @ (map f ys)
\end{verbatim}
\end{screen}
\subsection*{2}
\begin{verbatim}
map f (rev l) = rev (map f l)    
\end{verbatim}
を証明せよ.

\begin{screen}
\begin{verbatim}
(1) l = [] のとき
map f (rev []) = map f [] = [] = rev[] = rev (map f [])

(2) lが成り立つと仮定して l = x :: xsを示す.
map f (rev (x::xs)) = map f (rev xs @ x)  (revの定義)
                    = map f (rev xs) @ map f [x] (1の証明より)
                    = map f (rev xs) @ [f x]     (mapを適用した)
                    = rev (map f xs) @ [f x]     (仮定から)
                    = rev (f x :: (map f xs))    (revの定義から)
                    = rev (map f (x::xs))        (mapの定義から)
以上より成り立つことが示された   
\end{verbatim}
\end{screen}

\subsection*{3}
任意のリストxs, ys, zsに対して
\begin{verbatim}
revapp (xs @ ys) zs = revapp ys (revapp xs zs)    
\end{verbatim}
が成り立つことを証明せよ.
\begin{screen}
\begin{verbatim}
(1) xs = [] のとき
revapp ([] @ ys) zs = revapp ys zs
                    = revapp ys ([] :: zs)
                    = revapp ys (revapp xs zs)

(2) xsのとき成り立つと仮定してx::xsが成り立つことを示す.
revapp ((x::xs)@ys) zs = revapp (xs @ ys) (x::zs)
                       = revapp ys (revapp xs (x::zs))
                       = revapp ys (revapp (x::xs zs)   
\end{verbatim}
\end{screen}
\subsection*{4}
任意の木tに対して
\begin{verbatim}
size (reflect t) = size t
\end{verbatim}
を示せ.
\begin{screen}
\begin{verbatim}
(1) P(Lf)が成り立つとする.
  size (reflect Lf) = size (Lf) = 0
(2) t1, t2の場合を仮定してt = Br(v, t1, t2)が成り立つ時を示す.
  size (reflect t) = size (reflect Br(v, t1, t2))
                   = size (Br(v, reflect t2, reflect t1))
                   = 1 + size t2 + size t1
                   = size t
\end{verbatim} 
\end{screen}
\section*{課題7.2}
\lstinputlisting[]{./kadai72.ml}
\begin{lstlisting}
# preprod [1;2;3;4];;
- : int = 24
# preprod [0; 1; 2;];;
- : int = 0
# preprod [1;2;3;0];;
Exception: Zero.
# preprod [1;2;3];;
- : int = 6 
\end{lstlisting}
\end{document}